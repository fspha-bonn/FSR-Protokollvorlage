% !TeX spellcheck = de_DE
% Die erste (unkommentierte) Zeile im Dokument legt immer die Dokumentklasse fest
\documentclass[ngerman,toc=sectionentrywithdots]{scrartcl}
\usepackage{FS-pakete}

% Stand: 01.07.2025

% Wichtige commands
%====================================================================
\DraftwatermarkOptions{stamp=true} % Wasserzeichen, true or false
\newcommand{\Datum}{00.00.2000}
\newcommand{\Vorsitz}{Vorsitz} % (Sitzungsleitung, üblicherweise Vorsitz) Unterschrift notwendig
\newcommand{\Protokollierung}{Protokoll} % Unterschrift notwendig
\newcommand{\Startzeit}{18:30} % Beginn der Sitzung
\newcommand{\Endzeit}{19:30} % Ende der Sitzung

% --- Wenn Leer wird es einfach ignoriert ---
\newcommand{\Translator}{} % Unterschrift nicht notwendig
\newcommand{\Untertitel}{}

%====================================================================

% Jetzt startet das eigentliche Dokument
\begin{document}
	\Anfang

	\TOP{Bestätigung FSR Protokolle}
	\Protokoll{deutsche}{01.01.2000}{Dafür}{Dagegen}{Enthaltung}

	\TOP{Berichte []}
	\begin{itemize}
		\lis{Bericht 1}
	\end{itemize}

	\TOP{Sonstige}
	\begin{itemize}
		\lis{Sonstiges 1}
	\end{itemize}
	
	
	% Beispiele:
	\abst{Fragestellung}{dafür}{dagegen}{enthaltung}{ungültig}
	\begin{abst-list}[lll]{Abstimmung}{Frage}
		Text:  & dafür & dagegen\\\hline
		Option1:  & Stimmen1 & A\\
		Option2:  & Stimmen2 & B
	\end{abst-list}
	\GOA[1]{Antrag}{Antragstext}

	%===================================================================
	\Ende
	\Unterschrift
	%\DCFSRC[Uhrzeit]{Datum} % Für Online-Sitzungen; Erstes Argument nur, falls die Uhrzeit abweicht
	\NFSR[Uhrzeit]{Datum}{Ort} % Erstes Argument nur, falls die Uhrzeit abweicht, ansonsten kann [Uhrzeit] gelöscht werden

	%===================================================================
	\Anwesenheit{
		Person 1,
		Person 2,}{}
	%\begin{Anwesenheit-simple}
	%	Person 1, Person 2
	%\end{Anwesenheit-simple}
	
	% Glossar nicht wirklich notwendig
	%\vfill
	%\textbf{Für eine Erklärung der Kürzel siehe im Glossar}
	%\Anhang{Glossar}

\begin{description}
	\item[AIfA:] Argelander-Institut für Astronomie
	\item[AStA:] Allgemeiner Studierendenausschuss
	\item[AVZ:] Allgemeines Verfügungszentrum, Endenicher Allee 11-13
	\item[AWD:] Anwesenheitsdienst
	\item[BK:] Berufungskommission
	\item[BKV:] Bücherkommissionsverkauf
	\item[BuFaTa:] Bundesfachschaftentagung
	%\item[DPG:] Deutsche Physikalische Gesellschaft
	\item[FK:] Fachkonferenz der Fachgruppe Physik/Astronomie
	\item[FKGO:] Geschäftsordnung der Fachschaftenkonferenz
	\item[FSK:] Fachschaftenkonferenz der Fachschaften der Uni Bonn oder\\
	Fachschaftenkollektiv
	\item[FSR:] Fachschaftsrat
	\item[FSV:] Fachschaftsvertretung
	\item[FSVV:] Fachschaftsvollversammlung
	%\item[FSWO:] Fachschaftswahlordnung
	\item[GOSAFK:] Geschäftsordnung- und Satzungsausschuss der Fachschaftenkonferenz
	\item[HISKP:] Helmholtz-Institut für Strahlen- und Kernphysik, Nussallee 14-16
	\item[HRZ:] Hochschulrechenzentrum
	\item[IAP:] Institut für angewandte Physik, Wegelerstr. 8
	\item[jDPG:] junge Deutsche Physikalische Gesellschaft
	\item[MatNat-FK:] Fachschaftenkonferenz der Fachschaften der mathematisch-naturwissenschaftlichen Fakultät
	\item[MPIfR:] Max-Planck Institut für Radioastronomie
	\item[OE:] Orientierungseinheit. Einführungsveranstaltung für Erstsemester.
	\item[PI:] Physikalisches Institut, Nussallee 12
	%\item[RFWU:] Rheinische Friedrich-Wilhelms-Universität Bonn
	\item[SP:] Studierendenparlament
	\item[TRA:] Transdisciplinary research area (Transdisziplinäres Forschungsgebiet)
	\item[ULB:] Universitäts- und Landesbibliothek
	\item[WPAF:] Wahlprüfungsausschuss der Fachschaftenkonferenz
	\item[WPHS/WPH:] Wolfgang-Paul-Hörsaal, Kreuzbergweg 28
	\item[ZaPF:] Zusammenkunft aller (deutschsprachigen) Physik Fachschaft. BuFaTa der Physik.
\end{description}
\end{document}

% --- Notwendig für jedes Protokoll: ---

% \Anfang               - Erstellt Titel, Inhaltsverzeichnis etc.
% \Protokoll[]{de/en}{Datum}{Dafür}{Dagegen}{enthaltung}
%  - Abstimmung für Annahme FSR Protokolle. Optionales Argument für Text nach der Abstimmung.
% \Ende                 - Erstellt einen Einzeiler mit Sitzungsende.
% \Unterschrift[6.5cm]  - Erstellt die Unterschriftsfelder. Optionales Argument macht das
%                         Unterschriftsfeld für Protokoll breiter, falls Name zu lang ist
% \Anwesenheit{Personen,}{}
% - Syntax SEHR wichtig, Argumente müssen mit ,} enden!
%   Für kleine Namensliste ins erste Argument Namen mit Komma separiert schreiben, zweites Argument leer lassen.
%   Für Namensliste als Anhang in zwei Spalten Namen in erstes und zweites Argument aufteilen.

%====================================================================
% --- Weitere Befehle ---

% \NFSR[Uhrzeit]{Datum}{Ort} - Wann/Wo die nächste Sitzung stattfindet
% \DCFSR[Uhrzeit]{Datum}     - Wenn die nächste Sitzung auf discord stattfindet

% \TOP{Titel}             - Erstellt einen TOP Abschnitt. Der counter Nr wird erhöht.
% \Anhang{Titel}          - Erstellt einen Anhang Abschnitt. Der counter Anh wird erhöht.
% \GOA[1]{Antrag}{Inhalt} - Erstellt eine Box für einen GO Antrag.
%                           Erstes Argument zu Anpassen der Breite in Einheiten von \textwidth.

% \abst{Frage}{Dafür}{Dagegen}{Enthaltung}{Ungültig}
% - Enthaltung oder Ungültig leer lassen, je nach Meinungsbild, Abstimmung oder geheime Wahl.
% \begin{abst-mult}[ll]{Abstimmung}{Frage}    Option & Stimmen\\    \end{abst-mo}
% - Abstimmungsumgebung für beliebige Art von Abstimmung

% - Abkürzungen
% \tit -> \textit
% \ttt -> \texttt
% \tbf -> \textbf
% \lis -> \item[]\textbf  % für Listen


% - Bonus
% \usepackage{../footducks}   % footnotes, but ducks
% \renewcommand{\ducksign}{Text} % fußnote bei den Unterschriften
% \marklessfootnote[label]{Text}%  % Fußnote ohne Referenz im Text, um manuell Fußnoten zu erstellen,
%                                   wo eigentlich keine möglich sind. (ggf. "}%" für spacing wichtig)
% \myfootref{label}  % Um manuelle Fußnoten zu referenzieren
% Umgebung Anwesenheit-simple für die kurze Version, erlaubt Fußnoten