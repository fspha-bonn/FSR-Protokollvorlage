% !TeX spellcheck = en_GB
\documentclass[UKenglish,toc=sectionentrywithdots]{scrartcl}
\usepackage{FS-pakete_en}
\usepackage{translations} % for quick translations

% Last updated: 01.07.2025

% Important commands
%============================================================
\DraftwatermarkOptions{stamp=true} % Watermark, true or false
\newcommand{\Date}{00.00.2000}
\newcommand{\Chair}{Chairperson} % Needs to sign
\newcommand{\MinutesKeeper}{Keeper of minutes} % Needs to sign
\newcommand{\Starttime}{18:30} % Start of meeting
\newcommand{\Endtime}{19:30} % End of meeting

% --- Ignored if left empty ---
\newcommand{\Translator}{} % Does not need to sign
\newcommand{\Subtitle}{}

%------------------------------------------------------------

% The important part starts here
\begin{document}
	\Begin
	%============================================================

	\AI{Confirmation of FSR-meeting minutes}
	\Minutes{German and English}{01.01.2000}{for}{against}{abst.}

	\AI{Reports []}
	\begin{itemize}
		\lis{Committee 1} Text
	\end{itemize}

	\AI{Other}
	\begin{itemize}
		\lis{Other 1} Text
	\end{itemize}
		
	%------------------------------------------------------------
		
	% Examples
	\poll{Question}{for}{against}{abstention}{invalid}
	\begin{poll-list}[lll]{Poll}{Question}
		Text:  & for & against\\\hline
		Option1: & Voting1 & A\\
		Option2: & Voting2 & B
	\end{poll-list}
	\RRP[1]{Request}{Content}
	
	%------------------------------------------------------------
	\End
	\Signs
	%============================================================
	
	%\DCFSR[Time]{Date} % For online meetings; First Argument only if time differs, then delete [Time]
	\NFSR{Date}{HISKP} % Optional Argument only if time differs
	\begin{Attendance}
		Person 1, Person 2,
	\end{Attendance}
	%============================================================
	
	% glossary not really needed
	%\vfill
	%\textbf{For an explanation of the abbreviations, see the glossary}
	%% !TeX spellcheck = en_GB
\Attach{Glossary}

\begin{description}
	\item[AIfA:] Argelander-Institute for Astronomy
	\item[AStA:] (General) Student's Union
	\item[AVZ:] Centre for general use, Endenicher Allee 11-13
	\item[AWD:] Attendance duty
	\item[BK:] Appointment committee
	\item[BKV:] Book comission sale
	\item[BuFaTa:] Federal Fachschafts' meeting
	\item[FK:] Department Meeting of physics and astronomy
	\item[FKGO:] Rules of procedure of the Fachschafts' conference
	\item[FSK:] Fachschafts' Conference or\\
	Fachschafts' Collective
	\item[FSR:] Fachschafts' Council
	\item[FSV:] Student assocation representatives
	\item[FSVV:] (Full) Student Body Council
	\item[FSWO:] Fachschafts Election Rules
	\item[GOSAFK:] Rules and Bylaws Committee of the Fachschafts' Conference
	\item[HISKP:] Helmholtz Institute for Radiation and Nuclear Physics, Nussallee 14-16
	\item[HRZ:] University Computing Centre
	\item[IAP:] Institute for Applied Physics, Wegelerstr. 8
	\item[MatNat-FK:] Fachschafts' Conference of the mathematics and natural science department
	\item[MPIfR:] Max-Planck Institute for Radioastronomy
	\item[OE:] Welcome Week. Introductory event for first years.
	\item[PI:] Institute of Physics, Nussallee 12
	\item[RFWU:] Rheinische Friedrich-Wilhelms-University of Bonn
	\item[SP:] Student Parliament
	\item[TRA:] Transdisciplinary research area
	\item[ULB:] University and State Library
	\item[WPAF:] Election Review Committee of the Fachschafts' Conference
	\item[WPHS/WPH:] Wolfgang Paul Lecture Hall, Kreuzbergweg 28
	\item[ZaPF:] Meeting of all (German speaking) physics Fachschafts. BuFaTa of physics.
\end{description}
\end{document}

%============================================================
% --- Necessary for every Minutes: ---

% \Begin           - Creates title, toc etc.
% \Minutes[]{de/en}{01.01.2000}{For}{Against}{Abstention}
%  - Poll for adopting minutes. Optional Argument for text after poll.
% \End             - Creates one line of text with end time
% \Signs[7cm]      - Crates the fields to sign. Optional argument changes the size
%                    of the field for the keeper of minutes, if the name is too long
% \begin{Attendance}
%   Namen...
% \end{Attendance}

%------------------------------------------------------------
% --- Other commands ---

% \NFSR[Time]{Date}{Place}  - When/where the next meeting takes place
% \DCFSR[Time]{Date}        - If the next meeting is online on discord

% \AI{Title}                - Creates an AI segment. The counter Nr is increased.
% \Attach{Title}            - Creates an attachment segment. The counter Anh is increased.
% \AttachPDF{Filename}		- Simply \includepdf[pages=-]{Filename}
% \RRP[1]{Request}{Content} - Creates a box for a request on the rules of procedures.
%                             First argument controls the width of the box, compared to \textwidth

% \poll{Question}{For}{Against}{Abstention}{Invalid}
% - Leave abstention or invalid empty depending on opinion poll, normal poll or secret poll.
% \begin{poll-list}[ll]{Poll}{Question}
%   Option & Votes\\
% \end{poll-list}
% - List environment for polls, to create any poll

% - Shorthands
% \tit -> \textit
% \ttt -> \texttt
% \tbf -> \textbf
% \lis -> \item[]\textbf  % for lists. Some compilers may not like line breaks between \lis and the text.

%============================================================
% --- Translations ---

% Translations can be used as \WORD[abbrev.]. If no optional argument is given
% the full translation is returned, else it will be abbreviated.
% E.g.:
% \FSR   -> Fachschafts' Council
% \FSR[] -> FC

% The following translations exist:
% FSR, FSV, FSVV, AK, FSK, Fachgruppe, FK, BK, AStA